% 本原稿用の条件マクロ
%章ごとにコンパイルできるようにするための設定.
%このマクロが定義されていない場合,チャプタ内は個別のTEXソースとして扱われる.
\expandafter\ifx\csname MasterFile\endcsname\relax
\documentclass[a4j,twoside,12pt]{thesis} % 修論・卒論など (ページが右端にでる)   

\usepackage{mysettings}
\usepackage{url}
\usepackage{bm}
\usepackage[table,xcdraw]{xcolor}
\usepackage{multirow}
\usepackage[subrefformat=parens]{subcaption}
\usepackage[dvipdfmx]{graphicx}
\usepackage{tabularx}
\usepackage{diagbox} 

\newcommand{\bhline}[1]{\noalign{\hrule height #1}}  
\newcommand{\bvline}[1]{\vrule width #1}  
\newcolumntype{Y}{&gt;{\centering\arraybackslash}X} %中央揃え

\captionsetup{labelsep = quad}
\captionsetup{subrefformat = parens}
\captionsetup{compatibility=false}
\begin{document}

\addtocounter{chapter}{+3}

\setlength{\baselineskip}{1.95zw}
\setlength{\textheight}{30\baselineskip}
\mainmatter

\fi
% これより上は削除しちゃダメ
% 本原稿用の条件マクロここまで
\renewcommand\thefootnote{\arabic{footnote})}


\chapter{実験}\label{exper}
% ここに本文
\section{実験環境}
実験環境は以下のとおりである.
\subsection*{ハードウェア}
\begin{itemize}
  \item CPU: AMD Ryzen9 5900X(12コア24スレッド)
  \item メモリ: 32GB
  \item GPU: RTX 3090
  \item GPU メモリ: 24GB
\end{itemize}
\subsection*{ソフトウェア}
\begin{itemize}
  \item OS: Ubuntu 20.04
  \item CUDA: 11.2
  \item cuDNN: 8.1
  \item PyTorch: 1.10.1
\end{itemize}

\section{データセット}
実験には Sentences Involving Compositional Knowledge (SICK) \cite{beltagy:arxiv15} データセットと Semantic Textual Similarity Benchmark (STS-b) \cite{cer-etal-2017-semeval} データセットを用いた.
それぞれのデータセットには文のペアとその類似度が 0-5 の範囲で含まれている.それぞれのデータセットの詳細は以下の表\ref{table:dataset}の通りである.
\begin{table}
  \caption{データセットの詳細}
  \label{table:dataset}
  \centering
  \begin{tabular}{l|rrrr}
    \bhline{1pt}
          & \multicolumn{1}{c}{train} & \multicolumn{1}{c}{test} & \multicolumn{1}{c}{validation} & \multicolumn{1}{c}{total} \\
    \hline
    SICK  & 4,439                     & 4,906                    & 495                            & 9,840                     \\
    STS-b & 5,749                     & 1,379                    & 1,500                          & 8,628                     \\
    \bhline{1pt}
  \end{tabular}
\end{table}

\section{グラフの作成}
この節では, グラフの作成手法について述べる.

\subsection{ノード生成}
各ノードはそれぞれの単語とし, ノードを生成した.
各ノードの特徴量は,対応する単語のレンマをもとに, 事前学習済み fastText\cite{bojanowski2017enriching} モデルから得られるものとした.
ここで, 事前学習済みモデルには Common Crawl データセットでの事前学習済みモデルを利用した.
本研究では, 未知語に対応するために Word2Vec ではなく fastText を利用した.

\subsection{エッジ生成}
エッジの生成方法では,2 種類の方法について以下のように実験を行った.
\subsubsection{依存関係に基づくエッジ生成}
依存関係に基づくエッジ生成には, Stanza\cite{qi2020stanza} ライブラリを利用した Universal Dependencies に基づく係り受け解析を用いた.
係り受け解析の結果から, 単語間に依存関係が存在した場合に双方向エッジを作成した.

\subsubsection{依存関係と隣接関係に基づくエッジ生成}
依存関係と隣接関係に基づくエッジ生成には, 依存関係に基づくエッジに加えて, 単語間に隣接関係があれば双方向エッジを作成した.

\section{ハイパーパラメータ}
この節ではモデルのハイパーパラメータについて説明する.
Graph Convolutional Network (GCN) \cite{kipf2017semi} , または GraphSAGE \cite{hamilton2017inductive} の層数は 3 層とし, GCN の第 1 層, 第 2 層, 第 3 層の出力次元はそれぞれ 128, 64, 32 とした. また, NTN 層では $K=16$ とした.

\section{評価指標}
モデルの性能評価にはピアソンの相関係数を用いた.
\section{ベースライン}
ベースラインには以下の3つを用いた.
1 つ目が Elvys らによる, 長さ 5 の局所的文脈を考慮した Siamese LSTM (Elvys's Siamese LSTM),
2 つ目が TextSimGNN\cite{zhou2020sentence}でもベースラインとして用いられている,畳み込み層, Max プーリング層, Flatten 層からなるモデル (SemEval-CNN) ,
3 つ目が TextSIMGNN のグラフ構築法 2a (STGCM2a) , 4 つ目が TextSIMGNN のグラフ構築法 3 (STGCM3) である.

\section{実験結果}
実験結果を表\ref{table:result}, ベースラインの実験結果を表\ref{table:baseline}に示す.
表\ref{table:result}の 2 行目では, 依存関係のみに基づくグラフ構築法を UD, 依存関係と隣接関係に基づくグラフ構築法を UD-adjacent と表記している.
また, 表\ref{table:result}の 1 列目の略称は, 第\ref{meth}章で示したそれぞれの手法の略称と対応している.
\begin{table}
  \caption{実験結果}
  \label{table:result}
  \begin{center}
  \scalebox{0.85}{
  \begin{tabular}{p{35mm}|rrrr}
    \bhline{1pt}
          & \multicolumn{2}{c}{SICK}  & \multicolumn{2}{c}{STS-b} \\
          & \multicolumn{1}{c}{UD} & \multicolumn{1}{c}{UD-adjacent} & \multicolumn{1}{c}{UD} & \multicolumn{1}{c}{UD-adjacent} \\
    \hline
    GCN-Att-NTN & $\mathbf{0.848 \pm 0.014}$ & $\mathbf{0.829 \pm 0.019}$ & $\mathbf{0.419 \pm 0.052}$ & $\mathbf{0.406 \pm 0.004}$ \\
    SAGE-Att-NTN     & $0.830 \pm 0.020$ & $0.829 \pm 0.019$ & $0.389 \pm 0.042$ & $0.387 \pm 0.054$ \\
    GCN-SAGPool-NTN  & $0.767 \pm 0.022$ & $0.732 \pm 0.037$ & $0.312 \pm 0.067$ & $0.351 \pm 0.095$ \\
    SAGE-SAGPool-NTN & $0.700 \pm 0.041$ & $0.723 \pm 0.047$ & $0.219 \pm 0.044$ & $0.270 \pm 0.046$ \\
    GCN-Att-cos      & $0.754 \pm 0.008$ & $0.754 \pm 0.011$ & $0.400 \pm 0.013$ & $0.398 \pm 0.011$ \\
    SAGE-Att-cos     & $0.751 \pm 0.010$ & $0.744 \pm 0.010$ & $0.347 \pm 0.020$ & $0.343 \pm 0.021$ \\
    GCN-SAGPool-cos  & $0.462 \pm 0.045$ & $0.581 \pm 0.040$ & $0.315 \pm 0.025$ & $0.245 \pm 0.021$ \\
    SAGE-SAGPool-cos & $0.586 \pm 0.041$ & $0.498 \pm 0.025$ & $0.321 \pm 0.017$ & $0.309 \pm 0.017$ \\
    \bhline{1pt}
  \end{tabular}
  }
  \end{center}
\end{table}

\begin{table}
  \caption{ベースラインの実験結果}
  \label{table:baseline}
  \begin{center}
  \begin{tabular}{p{35mm}|rr}
    \bhline{1pt}
          & \multicolumn{1}{c}{SICK} & \multicolumn{1}{c}{STS-b} \\
    \hline
    Elvys's Siamese LSTM &      $0.855$   &          $0.758$     \\
    SemEval-CNN &     $0.855 \pm 0.015$   &       $0.776 \pm 0.015$     \\
    TextSimGNN (STGCM2a) &  $0.754 \pm 0.010$     & $0.352 \pm 0.010 $        \\
    TextSIMGNN (STGCM3) &   $0.428 \pm 0.010$ &   $0.786 \pm 0.010 $   \\
    \bhline{1pt}
  \end{tabular}
  \end{center}
\end{table}
% 本原稿用の条件マクロ
% これ以降は削除しちゃダメ
\expandafter\ifx\csname MasterFile\endcsname\relax
\def\MasterFile{本原稿です}

% 参考文献
% % 本原稿用の条件マクロ
%章ごとにコンパイルできるようにするための設定.
%このマクロが定義されていない場合,チャプター内は個別のTEXソースとして扱われる.
\expandafter\ifx\csname MasterFile\endcsname\relax
\documentclass[a4j,12pt]{thesis} % 修論・卒論など (ページが右端にでる)   
\usepackage{mysettings}
\usepackage{url}

\begin{document}

\setlength{\baselineskip}{1.95zw}
\setlength{\textheight}{30\baselineskip}
\backmatter

\fi
% これより上は削除しちゃダメ
% 本原稿用の条件マクロここまで

%参考文献

\bibliographystyle{sieicej}
\bibliography{thesisB}

\clearpage


% 本原稿用の条件マクロ
% これ以降は削除しちゃダメ
\expandafter\ifx\csname MasterFile\endcsname\relax
\def\MasterFile{本原稿です}
\end{document}
\fi
% 本原稿用の条件マクロここまで


\bibliographystyle{junsrt}
\bibliography{thesisB}

\end{document}
\fi
% 本原稿用の条件マクロここまで
