% 本原稿用の条件マクロ
%章ごとにコンパイルできるようにするための設定.
%このマクロが定義されていない場合,チャプタ内は個別のTEXソースとして扱われる.
\expandafter\ifx\csname MasterFile\endcsname\relax
\documentclass[a4j,twoside,12pt]{thesis} % 修論・卒論など (ページが右端にでる)   

\usepackage{mysettings}
\usepackage{url}
\usepackage{bm}
\usepackage[table,xcdraw]{xcolor}
\usepackage{multirow}
\usepackage[subrefformat=parens]{subcaption}
\usepackage[dvipdfmx]{graphicx}
\usepackage{tabularx}
\usepackage{diagbox} 

\newcolumntype{Y}{&gt;{\centering\arraybackslash}X} %中央揃え

\captionsetup{labelsep = quad}
\captionsetup{subrefformat = parens}
\captionsetup{compatibility=false}
\begin{document}

\addtocounter{chapter}{+3}

\setlength{\baselineskip}{1.95zw}
\setlength{\textheight}{30\baselineskip}
\mainmatter

\fi
% これより上は削除しちゃダメ
% 本原稿用の条件マクロここまで
\renewcommand\thefootnote{\arabic{footnote})}


\chapter{実験}\label{exper}
% ここに本文
適宜 section 分けして書く

% 本原稿用の条件マクロ
% これ以降は削除しちゃダメ
\expandafter\ifx\csname MasterFile\endcsname\relax
\def\MasterFile{本原稿です}

% 参考文献
% % 本原稿用の条件マクロ
%章ごとにコンパイルできるようにするための設定.
%このマクロが定義されていない場合,チャプター内は個別のTEXソースとして扱われる.
\expandafter\ifx\csname MasterFile\endcsname\relax
\documentclass[a4j,12pt]{thesis} % 修論・卒論など (ページが右端にでる)   
\usepackage{mysettings}
\usepackage{url}

\begin{document}

\setlength{\baselineskip}{1.95zw}
\setlength{\textheight}{30\baselineskip}
\backmatter

\fi
% これより上は削除しちゃダメ
% 本原稿用の条件マクロここまで

%参考文献

\bibliographystyle{junsrt}
\bibliography{thesisB}

\clearpage


% 本原稿用の条件マクロ
% これ以降は削除しちゃダメ
\expandafter\ifx\csname MasterFile\endcsname\relax
\def\MasterFile{本原稿です}
\end{document}
\fi
% 本原稿用の条件マクロここまで


\bibliographystyle{sieicej}
\bibliography{thesisB}

\end{document}
\fi
% 本原稿用の条件マクロここまで
