% 本原稿用の条件マクロ
%章ごとにコンパイルできるようにするための設定.
%このマクロが定義されていない場合,チャプター内は個別のTEXソースとして扱われる.
\expandafter\ifx\csname MasterFile\endcsname\relax
\documentclass[a4j,twoside,12pt]{thesis} % 修論・卒論など (ページが右端にでる)   
\usepackage{mysettings}
\usepackage{url}
%\usepackage{comment}
\usepackage{bm}
\begin{document}
\addtocounter{chapter}{+5}

\setlength{\baselineskip}{1.95zw}
\setlength{\textheight}{30\baselineskip}

\fi
% これより上は削除しちゃダメ
% 本原稿用の条件マクロここまで


\chapter{おわりに}\label{conc}
% ここに本文
本研究では, 文の意味的類似度評価におけるグラフ構造の利用手法を提案した.
実験は, 文からのグラフの作成, ノードの畳み込み, グラフを表現する埋め込み生成, グラフ間の類似度の評価という流れで行い, それぞれの過程で 2 通りずつの方法を採用し,計16 通りの方法を試した.
16 通りの方法のうち, グラフの作成には単語の依存関係に基づく構築法を, ノードの畳み込みにはGCN を, グラフを表現する埋め込みの生成にはアテンションモジュールを, グラフ間の類似度評価にはニューラルテンソルネットワークをそれぞれ利用した方法において最も優れた結果が得られた. この方法では Sentence Involving Compositional Knowledge データセット で, ピアソンの相関係数0.848 ±0.014 という結果が得られた.
これは Elvys らの結果と同等であり, このことは STS タスクにおけるグラフ構造の利用の有効性を示唆している.
\par 将来の展望としては, 単語の埋め込み生成に文脈を考慮したモデルを用いることで, 今回のモデルの欠点を補うことが期待できる.

% 本原稿用の条件マクロ
% これ以降は削除しちゃダメ
\expandafter\ifx\csname MasterFile\endcsname\relax
\def\MasterFile{本原稿です}

% 参考文献
% % 本原稿用の条件マクロ
%章ごとにコンパイルできるようにするための設定.
%このマクロが定義されていない場合,チャプター内は個別のTEXソースとして扱われる.
\expandafter\ifx\csname MasterFile\endcsname\relax
\documentclass[a4j,12pt]{thesis} % 修論・卒論など (ページが右端にでる)   
\usepackage{mysettings}
\usepackage{url}

\begin{document}

\setlength{\baselineskip}{1.95zw}
\setlength{\textheight}{30\baselineskip}
\backmatter

\fi
% これより上は削除しちゃダメ
% 本原稿用の条件マクロここまで

%参考文献

\bibliographystyle{sieicej}
\bibliography{thesisB}

\clearpage


% 本原稿用の条件マクロ
% これ以降は削除しちゃダメ
\expandafter\ifx\csname MasterFile\endcsname\relax
\def\MasterFile{本原稿です}
\end{document}
\fi
% 本原稿用の条件マクロここまで


\bibliographystyle{junsrt}
\bibliography{thesisB}

\end{document}
\fi
% 本原稿用の条件マクロここまで
