\documentclass{abst}

\begin{document}

%%%%%%%%%%%%%%%%%%%%%%%%%%%%%%%%%%%%%%%%%%%%%%%%%%%%%%%%%%%%%%%%%%%
\研究室名{外山}
\氏名{加~~藤~~~辰~~弥}
\卒論題目{%
文の意味的類似度評価におけるグラフ構造の利用
}

%%%%%%%%%%%%%%%%%%%%%%%%%%%%%%%%%%%%%%%%%%%%%%%%%%%%%%%%%%%%%%%%%%%
\卒論要旨{%
文の意味的類似度評価タスク(STS)とは, 2 文の類似度を求めるタスクである.
具体的には``Kids in red shirts are playing in the leaves''と``Children in red shirts are playing in the leaves''は類似度が高いと評価し, ``A woman is riding a horse''と``A man is opening a small package that contains headphones''は類似度が低いと評価する.2 文の類似度を求めることができると, 翻訳の性能評価であったり迷惑メール$\cdot$スミッシングなどの検出に役立つ.
\par 文間類似度の計算には, Grigori らにより編集距離を用いる手法が提案された.この手法では文の構造情報を扱っている.
その後, ニューラルネットワークが発展するにつれて, LSTM や RNN などの自然言語処理モデルを用いて文間類似度計算の研究が進められてきた.
これらのモデルで表される文はシーケンス構造である.
SemEvalで設定された STS タスクでは, Elvys らにより行われた LSTM を用いた先行研究で, ピアソンの相関係数 0.8549 という良好な結果を示している.
しかし, 一般的な自然言語処理モデルで表されるシーケンス構造には, 構造情報が欠落しているが, 構造情報には依存関係のようなグラフ構造で表現する事に適した情報が豊富に存在する.
このような情報を特徴量として扱うことで, 文の表現をより豊かにできるのではないかと考えた.
\par したがって, 構造情報の面からのアプローチとして, グラフ構造と, それを処理するためのニューラルネットワークであるグラフニューラルネットワーク (GNN) の利用を検討した.
代表的な GNN として, Graph Convolutional Network (GCN) や Graph Attention Network (GAT), GraphSAGE など様々な手法が提案されている.また, グラフ編集距離を近似的に求める SimGNN や, GCN を用いたグラフプーリング手法である Self-Attention Graph Pooling (SAGPool) といった手法も提案されている.
\par 本研究の目的は, 意味的類似度評価タスクにおいて, グラフ構造を用いたアプローチの有効性を明らかにすることである.
\par 実験は, 文からのグラフの生成, ノードの畳み込み, グラフを表現する埋め込み生成, グラフ間の類似度の評価という流れで行い, それぞれの過程で 2 通りずつの方法を採用し,計 16 通りの方法を試した.
文からのグラフの生成方法には, 単語間の依存関係のみに基づく生成方法と, 単語間の依存関係と隣接関係に基づく生成方法を用いた.
実験の評価方法としてはピアソンの相関係数を算出し, モデルの性能を測った.
データセットには, 短文データセットである Sentences Involving Compositional Knowledge (SICK) と混在長データセットである STS Benchmark (STS-b) の二つのデータセットを用いた.
\par 今回の実験では, 依存関係に基づくグラフの構築法を用い, モデルは SimGNN と同様のモデルを利用した方法で最も優れた結果が得られた.
このモデルでは SICK で Elvys らの結果と同等程度の良好な性能を示した.このことは STS タスクにおけるグラフ構造の利用の有効性を示唆している.
しかし, STS-b では十分な結果を得られなかった.
この理由として 二つの理由が推測された.
一つ目の理由として, 生成したグラフには構造情報は含まれていたが, 意味情報が十分に含まれていなかったことが挙げられる.
二つ目の理由として, グラフサイズに差が生まれることが挙げられる.
これらの理由から生じるデータの差が今回の実験の結果を導いたと考えられる.
\par 将来の展望としては, 単語の埋め込み生成に文脈を考慮したモデルを用いることで, 今回のモデルの欠点を補うことができるのではないかと考える.
}

\end{document}
