\documentclass{abst}

\begin{document}

%%%%%%%%%%%%%%%%%%%%%%%%%%%%%%%%%%%%%%%%%%%%%%%%%%%%%%%%%%%%%%%%%%%
\研究室名{外山}
\氏名{加~~藤~~~辰~~弥}
\卒論題目{%
文の意味的類似度評価におけるグラフ構造の利用
}

%%%%%%%%%%%%%%%%%%%%%%%%%%%%%%%%%%%%%%%%%%%%%%%%%%%%%%%%%%%%%%%%%%%
\卒論要旨{%
文の意味的類似度評価タスク(STS)とは,2 文の類似度を求めるタスクである.
具体的には``Kids in red shirts are playing in the leaves''と``Children in red shirts are playing in the leaves''は類似度が高いと評価し,``A woman is riding a horse''と``A man is opening a small package that contains headphones''は類似度が低いと評価する.
\par STS タスクにおいて,Elvys らにより行われた LSTM を用いた先行研究では,ピアソンの相関係数 0.8549 という良好な結果を示している.
しかし,LSTM や RNN などの一般的な自然言語処理モデルで表されるシーケンス構造には,構造情報が欠落しているが,構造情報には依存関係,グラフ構造で表現する事に適した情報が豊富に存在する.
このような情報を特徴量として扱うことで,文の表現をより豊かにできるのではないかと考えた.
\par したがって,構造情報の面からのアプローチとして,グラフ構造と,それを処理するためのニューラルネットワークであるグラフニューラルネットワーク(GNN)の利用を検討した.
代表的な GNN として,Graph Convolutional Network (GCN) や Graph Attention Network (GAT), GraphSAGE など様々な手法が提案されている.
\par 本研究の目的は,意味的類似度評価タスクにおいて,グラフ構造を用いたアプローチの有効性を明らかにすることである.
\par 実験では,文からのグラフの生成に2通り, ノードの畳み込みに2通り, グラフを表現する埋め込みに2通り, グラフ間の類似度の評価に2通りの計16通りの方法で実験を行った.
% \par 実験ではまず,文からグラフを生成した.
% 文からのグラフの生成方法では,単語間の依存関係のみに基づく生成方法と,単語間の依存関係と隣接関係に基づく生成方法の 2 通りの方法で比較した.
% その後,ノード間の畳込みによりノードレベルの埋め込みを生成した.
% ノード間の畳込みには,SimGNN で用いられている GCN と,GraphSAGE の 2 通りの方法で比較した.
% その後,グラフを表現する埋め込みを生成した.
% グラフを表現する埋め込みの生成には,Attention Module を用いる方法と Self Attention Graph Pooling を用いる方法の2通りの方法で比較した.
% 最終的なグラフ間の類似度の計算はコサイン類似度を用いる方法と,Neural Tensor Network を用いる方法の 2 通りの方法で比較した.
実験の評価方法としてはピアソンの相関係数を算出し,モデルの性能を測った.
実験には短文データセットである Sentences Involving Compositional Knowledge(SICK)と混在長データセットである STS Benchmark(STS-b)の二つのデータセットを用いた.
\par 今回の実験では, SICK でモデルは Elvys らの結果と同等程度の良好な性能を示した.このことは STS タスクにおけるグラフ構造の利用の有効性を示唆している.
しかし,STS-b では十分な結果を得られなかった.
この理由として 二つの理由が推測された.一つ目の理由として,生成したグラフには構造情報は含まれていたが,意味情報が十分に含まれていなかったことが挙げられる.SICK では類似した文には同じ単語が多く含まれていたが,STS-b では,同じ単語が多く含まれていても類似度が低いことがある.
二つ目の理由として,グラフサイズに差が生まれることである.SICK は主に短文が含まれるのに対し,STS-b では混在長の文が含まれている.このことにより生じる.
これらの理由から生じるデータの差が今回の実験の結果を導いたと考えられる.
\par 将来の展望としては,単語の埋め込み生成に文脈を考慮したモデルを用いることで,SICK で Elvys らの結果を上回る可能性があり,STS-b においてもグラフ構造の有効性が示される可能性がある.また,論文内で扱ったグラフを扱う手法について,本研究では STS タスクに関して実験したが,他のタスクにおいても有効であるかを検証したい.
}

\end{document}
