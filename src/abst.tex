\documentclass{abst}

\begin{document}

%%%%%%%%%%%%%%%%%%%%%%%%%%%%%%%%%%%%%%%%%%%%%%%%%%%%%%%%%%%%%%%%%%%
\研究室名{外山}
\氏名{加~~藤~~~辰~~弥}
\卒論題目{%
文の意味的類似度評価におけるグラフ構造の利用
}

%%%%%%%%%%%%%%%%%%%%%%%%%%%%%%%%%%%%%%%%%%%%%%%%%%%%%%%%%%%%%%%%%%%
\卒論要旨{%
2 文の意味的類似度を求めることは, 翻訳の性能評価であったり, 迷惑メール$\cdot$スミッシングなどの検出に役立つ.
2 文の意味的類似度を求めるタスクとして, 文の意味的類似度評価 (STS) タスクがある.
このタスクは, 具体的には``Kids in red shirts are playing in the leaves''と``Children in red shirts are playing in the leaves''は意味的類似度が高いと評価し, ``A woman is riding a horse''と``A man is opening a small package that contains headphones''は意味的類似度が低いと評価するタスクである.
\par 文の類似度計算には, Grigori らにより編集距離を用いる手法が提案された.この手法では, 文の構造情報を扱っているが意味情報は考慮していない.
その後, ニューラルネットワークが発展するにつれて, LSTM や RNN などの自然言語処理モデルを用いた STS タスクに関する研究が進められてきた.
SemEval で設定された STS タスクでは, Elvys らにより行われた LSTM を用いた先行研究で, ピアソンの相関係数 0.8549 という良好な結果を示している.
しかし, LSTM や RNN などの一般的な自然言語処理モデルで扱われるシーケンスデータには, 構造情報が欠落している. 構造情報には依存関係のようなグラフ構造で表現する事に適した情報が存在する.
これらから, 意味情報だけでなく構造情報も扱うことで, 文を表す特徴量を増やすことができるのではないかと考えた.
\par したがって, 構造情報の面からのアプローチとして, グラフ構造と, それを処理するためのニューラルネットワークであるグラフニューラルネットワーク (GNN) の利用を検討した.
代表的な GNN として, Graph Convolutional Network (GCN) や Graph Attention Network (GAT), GraphSAGE といった手法がある.また, グラフ編集距離を近似的に求める SimGNN や, GCN を用いたグラフプーリング手法である Self-Attention Graph Pooling (SAGPool) のような手法も提案されている.
\par 本研究の目的は, STS タスクにおいて, グラフ構造を用いたアプローチの有効性を明らかにすることである.
\par 実験は, 文からのグラフの作成, ノードの畳み込み, グラフを表現する埋め込み生成, グラフ間の類似度の評価という流れで行い, それぞれの過程で 2 通りずつの方法を採用し, 計 16 通りの方法を試した.
実験の評価にはピアソンの相関係数を利用した.
データセットには, 短文データセットである Sentences Involving Compositional Knowledge (SICK) と混在長データセットである STS Benchmark (STS-b) の二つのデータセットを用いた.
\par 今回の実験で用いた 16 通りの方法のうち, グラフの作成には単語の依存関係に基づく構築法を, ノードの畳み込みには GCN を, グラフを表現する埋め込みの生成にはアテンションモジュールを, グラフ間の類似度評価にはニューラルテンソルネットワークを利用した方法で最も優れた結果が得られた.
この方法では SICK で Elvys らの結果と同等の良好な性能を示した.このことは STS タスクにおけるグラフ構造の利用の有効性を示唆している.
しかし, STS-b では十分な結果を得られなかった.
この理由として 二つの理由が推測された.
一つ目の理由として, 作成したグラフには構造情報は含まれていたが, 意味情報が十分に含まれていなかったことが挙げられる.
二つ目の理由として, グラフサイズに差が生まれることが挙げられる.
これらの理由から生じるデータの差が今回の実験の結果を導いたと考えられる.
\par 将来の展望としては, 単語の埋め込み生成に文脈を考慮したモデルを用いることで, 今回のモデルの欠点を補うことができるのではないかと考える.
}

\end{document}
