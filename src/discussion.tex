% 本原稿用の条件マクロ
%章ごとにコンパイルできるようにするための設定.
%このマクロが定義されていない場合,チャプター内は個別のTEXソースとして扱われる.
\expandafter\ifx\csname MasterFile\endcsname\relax
\documentclass[a4j,twoside,12pt]{thesis} % 修論・卒論など (ページが右端にでる)   
\usepackage{mysettings}
\usepackage{url}
\usepackage{comment} 
\usepackage{bm}
%\usepackage{multirow}
\usepackage{colortbl}
\usepackage{ulem}
\usepackage[subrefformat=parens]{subcaption}

\begin{document}
\addtocounter{chapter}{+4}
\setlength{\baselineskip}{1.95zw}
\setlength{\textheight}{30\baselineskip}
\mainmatter

\fi
% これより上は削除しちゃダメ
% 本原稿用の条件マクロここまで

\chapter{考察}\label{dis}
% ここに本文
適宜 section 分けして書く


%====================================================================================

% 本原稿用の条件マクロ
% これ以降は削除しちゃダメ
\expandafter\ifx\csname MasterFile\endcsname\relax
\def\MasterFile{本原稿です}

% 参考文献
% % 本原稿用の条件マクロ
%章ごとにコンパイルできるようにするための設定.
%このマクロが定義されていない場合,チャプター内は個別のTEXソースとして扱われる.
\expandafter\ifx\csname MasterFile\endcsname\relax
\documentclass[a4j,12pt]{thesis} % 修論・卒論など (ページが右端にでる)   
\usepackage{mysettings}
\usepackage{url}

\begin{document}

\setlength{\baselineskip}{1.95zw}
\setlength{\textheight}{30\baselineskip}
\backmatter

\fi
% これより上は削除しちゃダメ
% 本原稿用の条件マクロここまで

%参考文献

\bibliographystyle{junsrt}
\bibliography{thesisB}

\clearpage


% 本原稿用の条件マクロ
% これ以降は削除しちゃダメ
\expandafter\ifx\csname MasterFile\endcsname\relax
\def\MasterFile{本原稿です}
\end{document}
\fi
% 本原稿用の条件マクロここまで

\bibliographystyle{sieicej}
%% %\bibliographystyle{junsrt}
\bibliography{thesisB}


\end{document}
\fi
% 本原稿用の条件マクロここまで
